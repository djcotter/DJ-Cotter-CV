\documentclass[]{kyvernitis-resume}
\fullname{Daniel J. Cotter}
% \jobtitle{Sofware Engineer}
\usepackage{xcolor}
\definecolor{navyblue}{RGB}{0, 83, 137}
\newcommand\tcb[1]{\textcolor{navyblue}{\textbf{#1}}}

\begin{document}
\resumeheader
{\phone{+1 480-717-8466}}
{\email{daniel.j.cotter@gmail.com}}
{\website{djcotter.github.io}}
{\linkedin{danjcotter}}
{\scholar{D.J. Cotter}}
{\github{djcotter}}

\begin{section}{Education}
    \begin{subsectionnobullet}{PhD in Genetics}{Doctor of Philosophy}{Stanford University}{Sep. 2023}
        \item{Dissertation: The effects of relatedness and sex-biased demographic processes on human genetic variation}
        \item{Advisor: Noah Rosenberg.}
        \item{Thesis Committee: Jonathan Pritchard, Stephen Montgomery, Hua Tang}
    \end{subsectionnobullet}
    
    \begin{subsectionnobullet}{BS in Biological Sciences}{Bachelor of Science}{Arizona State University}{May 2018}
        \item \vspace{-5mm}
    \end{subsectionnobullet}
\end{section}

\begin{section}{Experience}
    \begin{subsection}{Stanford University}{Graduate Student Research Assistant}{Sep. 2018 -- Present}{Stanford, CA}
        \item Developed \tcb{mathematical and statistical models} to explore patterns of genomic sharing and understand genetic relatedness. 
        \item Designed and implemented \tcb{statistical corrections} to account for variations in sample sizes when analyzing population-level data.
        \item Analyzed and interpreted \tcb{RNA-seq} data across tissues to understand gene-expression variation.
    \end{subsection}
    \vspace{-2mm}
     \begin{subsection}{University of California, San Francisco}{Amgen Scholar Summer Intern}{Summer 2016}{San Francisco, CA}
        \item Analyzed patterns of genomic sharing to understand the basis of \tcb{complex disease}.
        \item Conducted rigorous \tcb{quality control} procedures for genomic data from diverse populations.
    \end{subsection}
    \vspace{-2mm}
    \begin{subsection}{Arizona State University}{Undergraduate Research Assistant}{Oct. 2014 -- May 2018}{Tempe, AZ}
        \item Collaborated on \tcb{bioinformatics} projects using whole-genome sequencing data to explore genetic variation across global populations.
    \end{subsection}
\end{section}

\begin{section}{Publications}
\vspace{1mm}
\textbf{\underline{Peer-reviewed articles}}
\item \textbf{DJ Cotter}, EF Hofgard, J Novembre, Z Szpiech, NA Rosenberg (2023) \textit{A rarefaction approach for measuring population differences in rare and common variation.} Genetics, 224(2): iyad070.
\item \textbf{DJ Cotter}, AL Severson, S Carmi, NA Rosenberg (2022) \textit{Limiting distribution of X-chromosomal coalescence times under first-cousin consanguineous mating.} Theoretical Population Biology, 147, 1--15.
\item \textbf{DJ Cotter}, AL Severson, NA Rosenberg (2021) \textit{The effect of consanguinity on coalescence times on the X chromosome.} Theoretical Population Biology, 140: 32--43.
\item M Oliva, M Muñoz-Aguirre, S Kim-Hellmuth, V Wucher, ADH Gewirtz, \textbf{DJ Cotter}, \dots, SB Montgomery, \dots, BE Stranger (2020). \textit{The impact of sex on gene expression across human tissues.} Science, 369(6509).
\item ML Antonio*, Z Gao*, HM Moots*, \dots, \textbf{DJ Cotter}, \dots, R Pinhasi, JK Pritchard (2019). \textit{Ancient Rome: A genetic crossroads of Europe and the Mediterranean.} Science, 366: 708-714.
\item \textbf{DJ Cotter}*, SM Brotman*, MA Wilson Sayres (2016) \textit{Genetic Diversity on the Human X Chromosome Does Not Support a Strict Pseudoautosomal Boundary.} Genetics, 203(1): 485-492.


\textbf{\underline{Preprints (not peer-reviewed)}}
\item \textbf{DJ Cotter}, TH Webster, MA Wilson (2021) \textit{Genomic and demographic processes differentially influence genetic variation across the X chromosome.} bioRxiv, doi.org/10.1101/2021.01.31.429027.
\end{section}

\sectiontable{Technical Skills}{
    \entry{Programming and Math Languages}{R, Python, Mathematica, bash, shell scripting}
    \entry{Bioinformatics and Genomics}{SAMtools, bedtools, GATK, BWA}
    \entry{Data Analysis and Visualization}{Tidyverse, ggplot2, pandas, Matplotlib}
    \entry{High Performance Computing Tools}{Slurm, Snakemake}
    \entry{Version Control}{Git, GitHub}
}

\begin{section}{Teaching}
\begin{subsectionnobullet}{Introduction to Genetics, Ethics, and Society}{Instructor}{Spring 2022}{Stanford University}
    \item\vspace{-5mm}
\end{subsectionnobullet}
\begin{subsectionnobullet}{Introductory Python Programming for Genomics}{Teaching Assistant}{
Winter 2022}{Stanford University}
    \item\vspace{-5mm}
\end{subsectionnobullet}
\begin{subsectionnobullet}{Genomics}{Teaching Assistant}{
Winter 2021}{Stanford University}
    \item\vspace{-5mm}
\end{subsectionnobullet}
\begin{subsectionnobullet}{Freshman Seminar}{Teaching Assistant}{Fall 2015-Spring 2016}{Arizona State University}
\item \vspace{-5mm}
\end{subsectionnobullet}

\end{section}



\begin{section}{Presentations}
\item \textbf{DJ Cotter}, AL Severson, S Carmi, and NA Rosenberg. Population, Evolutionary, and Quantitative Genetics (2022). The effect of consanguinity on X-chromosomal and autosomal genomic sharing. (P)
\item \textbf{DJ Cotter}, AL Severson, and NA Rosenberg. American Society of Human Genetics Annual Meeting (2019). The effect of consanguinity on between-individual identity by descent on the X chromosome. (P)
\item \textbf{DJ Cotter}, TH Webster, and MA Wilson Sayres. American Society of Human Genetics Annual Meeting (2016). Genetic diversity across the pseudoautosomal boundary varies across human populations. (P)
\item \textbf{DJ Cotter}, TH Webster, and MA Wilson Sayres. Evolutionary Genomics of Sex (2016). Genetic diversity across the pseudoautosomal boundary varies across human populations. (P)
\item \textbf{DJ Cotter}, ML Spear, and RD Hernandez. UCSF SRTP Symposium (2016). Exploring identity by descent and population structure in African American and Latino individuals from Northern California. (O)
\item \textbf{DJ Cotter}*, SM Brotman*, and MA Wilson Sayres. Southern California Evolutionary Genetics and Genomics Meeting (2016). Genetic diversity on the human X chromosome does not support a strict pseudoautosomal boundary. (P)
\item \textbf{DJ Cotter}*, SM Brotman*, and MA Wilson Sayres. International Society for Evolution, Medicine \& Public Health (2015). Using genetic diversity to measure boundaries of the pseudoautosomal regions in human sex chromosomes. (P)
\item \textbf{DJ Cotter}*, SM Brotman*, and MA Wilson Sayres. SOLUR Undergraduate Research Symposium (2015). Using genetic diversity to measure boundaries of the pseudoautosomal regions in human sex chromosomes. (P)

\item \hfill (O=Oral, P=Poster)
\end{section}

% \sectiontable{Soft skills}{
%     \entry{Procrastination}{Procrastination at an expert level}
%     \entry{Avoiding Responsibility}{Outstanding ability to avoid meetings and responsibilities}
%     \entry{Sarcasm}{Fluent in sarcasm and irony}
% }

\sectiontable{Recognitions and Awards}{
    \entry{Graduate Fellow}{Stanford Center for Computational,\hfill \textit{2023}}
    \entry{}{Evolutionary and Human Genomics}
    \entry{Graduate Research Fellow}{National Science Foundation \hfill \textit{2018--2023}}
    \entry{Student of the Year}{School of Life Sciences, Arizona State University \hfill \textit{2018}}
}

\end{document}
